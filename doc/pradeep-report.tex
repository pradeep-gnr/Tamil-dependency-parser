\documentclass[11pt,letterpaper]{article}
\usepackage{fullpage}
\usepackage[pdftex]{graphicx}
\usepackage{amsfonts,eucal,amsbsy,amsopn,amsmath}
\usepackage{url}
\usepackage[sort&compress]{natbib}
\usepackage{natbibspacing}
\usepackage{latexsym}
\usepackage{wasysym} 
\usepackage{rotating}
\usepackage{fancyhdr}
\DeclareMathOperator*{\argmax}{argmax}
\DeclareMathOperator*{\argmin}{argmin}
\usepackage{sectsty}
\usepackage[dvipsnames,usenames]{color}
\usepackage{multicol}
\definecolor{orange}{rgb}{1,0.5,0}
\usepackage{multirow}
\usepackage{sidecap}
\usepackage{caption}
\renewcommand{\captionfont}{\small}
\setlength{\oddsidemargin}{-0.04cm}
\setlength{\textwidth}{16.59cm}
\setlength{\topmargin}{-0.04cm}
\setlength{\headheight}{0in}
\setlength{\headsep}{0in}
\setlength{\textheight}{22.94cm}
\allsectionsfont{\normalsize}
\newcommand{\ignore}[1]{}
\newenvironment{enumeratesquish}{\begin{list}{\addtocounter{enumi}{1}\arabic{enumi}.}{\setlength{\itemsep}{-0.25em}\setlength{\leftmargin}{1em}\addtolength{\leftmargin}{\labelsep}}}{\end{list}}
\newenvironment{itemizesquish}{\begin{list}{\setcounter{enumi}{0}\labelitemi}{\setlength{\itemsep}{-0.25em}\setlength{\labelwidth}{0.5em}\setlength{\leftmargin}{\labelwidth}\addtolength{\leftmargin}{\labelsep}}}{\end{list}}

\bibpunct{(}{)}{;}{a}{,}{,}
\newcommand{\nascomment}[1]{\textcolor{blue}{\textbf{[#1 --NAS]}}}


\pagestyle{fancy}
\lhead{}
\chead{}
\rhead{}
\lfoot{}
\cfoot{\thepage~of \pageref{lastpage}}
\rfoot{}
\renewcommand{\headrulewidth}{0pt}
\renewcommand{\footrulewidth}{0pt}


\title{11-712:  NLP Lab Report}
\author{Pradeep Prabakar Ravindran}
\date{April 26, 2013 (Final Due date)}

\begin{document}
\maketitle
\begin{abstract}
First draft of a report documenting my experience developing a dependency parser for Tamil.
\end{abstract}

%\nascomment{brief introduction}
TamilDep is a tool that performs dependency parsing of the Tamil Language. This tool would soon be open-sourced
and become publicly available at https://github.com/pradeep-gnr/TamilDep. Along with
the code for a dependency parser, a corpora of tamil sentences annotated with their dependency parse trees 
is also expected to be released. Since there are very
few resources publicly available for Tamil,  I hope that this software would prove useful to researchers working on Tamil natural
language processing.

\section{Basic Information about Tamil}

Tamil is a Dravidian language that is widely spoken in the Indian state of Tamil Nadu
and the north eastern part of Sri Lanka. Reports estimate that there are about
70 million native Tamil Speakers all over the world (\cite{Wiki}). In terms of history, Tamil is also one of the oldest languages in the world with works of Tamil
literature dating back to almost 2000 years ago (\cite{Wiki}).\\

Grammatically, Tamil has a lot of characteristics that make it pretty different
from the Germanic languages. For example, Tamil almost always follows a subject-object-verb
oriented form (\cite{ramasamy2011tamil}). For instance, consider the following sentence.

\begin{center}
\begin{verbatim}
                      Avan      angu       selgiran 
                      (He)     (there)     (going)  
\end{verbatim}
\end{center}

The sentence approximately translates to 'He is going there'. Sentences where inflected verb forms
occurring at the end of the sentence are relatively common in Tamil. Also, similar to languages like Czech, 
Tamil is relatively a free word order language. For example, the sentence that was discussed above 
could be expressed in multiple forms such as

\begin{verbatim}
                      angu     avan    selgiran 
                     (There)   (he)     (going)

                       selgiran  avan    angu
                       (going)   (he)   (there)
\end{verbatim}
All of these sentences are perfectly valid in terms of grammatical structure. 
Another key characteristic of Tamil sentences is that words could be agglutinative and sometimes it is very difficult to even define what a word is. 
There are a number of ways in which suffixes could be added to the noun and verb stems to produce highly inflected forms.

For instance, 'has gotten over' which is a phrase in English could be
expressed as a single word in Tamil. Also, there could be multiple such words that would approximately translate to the sample
phrase. Some such examples are discussed below.

\begin{verbatim}
                 1) Nadanthathu
                 2) Nadanthathuvittathu
                 3) Nadanthayittru
\end{verbatim}

All these words express the phrase 'has gotten over' and they are all highly inflected
forms of the root word 'Nada' which roughly translates to 'happening' or 'proceeding'.
Since these type of verb phrases are relatively common in Tamil, identifying the 'root' of the head
words of such phrases would be a challenge from a dependency parsing perspective.

\section{Past Work on the Syntax and NLP tools for Tamil}

One of the most infleuntial books on Tamil grammar written in English was by Thomas Lehmann
: A grammar of modern Tamil (\cite{lehmann1989grammar}). The book gives a very detailed analysis of several of
the morphological and syntactic phenomenon that characterize written Tamil. While languages like English do not differ significantly
from their spoken and written form, there are significant differences among the spoken and written forms in Tamil.
Tamil has the property that sentences could be semantically valid even when certain words are omitted and spoken 
Tamil usually consists of a lot of missing words. Written Tamil however is more morphologically richer than its spoken version.
Lehmann also notes that Tamil is relatively a head final language and that the phrasal head appears in several inflected forms at
the end of the phrase (\cite{lehmann1989grammar}). Also, another key characteristic of Tamil is that the verbs agree with gender, tense etc. 
One of the key differences in tamil syntax when compared to other languages is that the language exhibits
a restricted free word order. Hence there are a multitude of semantically and syntactically valid sentences that could
be formed using the same set of words. From a parsing perspective - this property poses a lot of challenges because unlike English, writing context free grammar rules to characterize Tamil Syntax is a non-trivial problem. (\cite{vidyapeethamrobust}) discuss the challenges
they face because of this issue while building a spoken dialog system for Tamil and turn to dependency parsing to aid semantic annotation
of sentences.
\\

Unlike other Indian languages like Hindi, Telugu and Bengali - Tamil does not have a lot of annotated corpora for NLP tasks.
The Tamil dependency treebank ((\cite{ramasamy2011tamil})) which was released recently is the only publicly available dependency corpora
that I could find online. The authors have also released an annotation manual that describes the rules that were used in
labeling dependency edges among words. The entire corpus was annotated in the Prague dependency treebank (PDT) (\cite{bohmova2003prague}) format and a detailed description
of this corpora and the annotation methodology that was used will be discussed in part 3. Although this corpus is quite small (about
3000 words), the annotations are quite diverse and cover various types of sentences in Tamil. The same group also used
this corpus to build a rule based and corpus based dependency parser and report a 75 \% accuracy when sentences are labeled with POS tags.\\

There was also an earlier paper (\cite{dhanalakshmi2010natural}) that used morphological analysis along with various custom heuristics for syntactic parsing
of Tamil to identify phrasal consituents. Besides the work done by Ramaswamy et al (\cite{ramasamy2011tamil}), I could find only one related paper that describes
a set of Natural language tools that were created specifically for Tamil. They have also released an open source Morphological analyzer
for Tamil but unfortunately there was no dependency tree annotated corpora or software that was available for download. To the best of my knowledge, TamilDep would be 
the first open source dependency parser written specifically for Tamil. 

\section{Available Resources for Tamil}

First of all to build a good annotated corpus, we need good quality sentences in the target language. 
The source text must be of high quality both in grammar and content. 
The content must also be diverse in the number of topics covered to make the parser more robust while handling different types of inputs. 
Tamil has a separate Wikipedia portal where there are quite a significant of documents that have been written exclusively in Tamil. 
There are also several other corpora that have been publicly
released for various Tamil NLP tasks such as English-Tamil machine translation (\cite{RaBoMorphologicalProcessing2012}), Tamil Wordnet etc (\cite{rajendran2002tamil} et al.). \\

The English-Tamil parallel corpora released by (\cite{ramasamy2011tamil}) consists of a total of around 169871 sentences that have been crawled
from various news articles discussing multiple topics. From this corpus, I have extracted random sentences and 
constructed corpora A and B. Corpora A consists of about 70 sentences with 1126 tokens and corpora B contains 70 sentences with 1185 tokens. Additionally I have also 
created corpora C, which could be used as training data if I decide to pursue a supervised approach to dependency parsing. Corpora C
consists of about 210 sentences and 3500 tokens. The data has been uploaded in the corpora folder in the root repository. \\

To aid the dependency annotation process, I decided to extend on the work done by (\cite{ramasamy2011tamil})
In their earlier paper on dependency parsing experiments for Tamil, (\cite{ramasamy2011tamil}) describe their experiences creating a detailed 
annotation corpora for Tamil. The annotations are in the Prague treebank format that has a three level layer based annotation at the syntactic and lexical level. 
They have also publicly released their annotation manual and rules to guide dependency annotation for Tamil. This manual is very comprehensive and documents several phenomena that is unique to Tamil
and issues that a person needs to consider while performing the annotation. Because of the highly inflected nature of the language,
morphological analysis must also be performed prior to syntactic annotation. The manual also discusses strategies for morphological
analysis and outlines the rules in detail so that it could be easily extended into a morphological analyzer. I decided to use the custom tag set 
that was used by (\cite{ramReport}) in the hope that future annotated corpora produced for Tamil would conform to a particular standard
and the research community can evaluate their methodologies on different corpora. Also, this would encourage the development
of supervised models which tend to work well when more data is available. 

\section{Survey of Phenomena in Tamil}

In this section, I shall describe some of the interesting phenomena that Tamil part of speech tags exhibit. Lehmann (\cite{lehmann1989grammar})
notes that Tamil has about eight part of speech tags (Nouns, Verbs, Postpositions, Adjectives, Adverbs, Quantifiers, Determiners,
and Conjunctions). Each of these part of POS tags exhibit some interesting behavior that help characterize Tamil language. 

\subsection{Nouns}

Lehmann states that nouns in Tamil are those words that can take case suffixes and also the suffixes (aay, aaka). The suffixes 
(aay, aaka) are adverbial suffixes that could be applied to several nouns. Lehmann also notes that not all nouns could take
these suffixes and these nouns are called as defective nouns. Nouns in Tamil could also be inflected in a number of ways according to
case and number. An inflected noun is usually of the form noun + (number) + case. The most common forms of suffixes that could inflect Tamil
nouns are given below.

\begin{enumerate}
 \item plural suffix 
 \item oblique suffix 
 \item euphonic suffix 
 \item case suffix
\end{enumerate}

Noun stems in Tamil are the stems of nouns as they would be listed in a dictionary. Tamil noun stems could both be simple and complex. 
Complex nouns are however formed by combining a root and a derivational suffix. For example, paal (milk) is a simple noun having no
suffixes while pati + ppu (study) is an example of a complex noun that has a verbal root (padi) and a suffix attached to it. 
Both these types are relatively common in Tamil. Tamil also exhibits a class of noun stem called as oblique stem that has no
meaning when it exists by itself but can combine with case suffixes and post positions to produce various noun forms.

\subsection{Pronouns}

Tamil has the interesting phenomena that verbs with gender and case markings can stand out as separate sentences even
when there is no explicit subject. As I discussed in the earlier sections, verb phrases in Tamil inflected with 
pronoun markers are relatively frequent in Tamil. For example, the verb 'walk' can be inflected as pronouns

\begin{enumerate}
 \item Nadanthan - (He walked)
 \item Nadandhal - (She walked)
 \item Nadandhadhu - (It walked)
\end{enumerate}

\subsection{Postpositions}

Postpositions are very important in Tamil and occur very frequently after nouns resulting in complex inflected phrases.
Lehmann notes that postpositions could be inflected or un-inflected nouns or they could also be non-finite verb forms. 
Lehmann also discusses 8 different scenarios in which postpositions occur and I have described some of them below. 

\begin{enumerate}
 \item Occurring after nouns (Nominative)- (kaadu + varai - (till the forest)))
 \item After nouns (accusative case) - (padippai + patri - (About your education))
 \item Occurring after nouns (dative case) - (Idharku + paatilaga (Instead of this))
 
\end{enumerate}

\subsection{Adjectives}
Lehmann notes that adjectives in Tamil usually are of two distinct types: simple and derived. Some examples of simple
and derived adjectives are given below.

\begin{enumerate}
 \item ketta paiyyan (bad boy)
 \item vayathana aasiriyar (old teacher) - Here the adjective 'vayathana' consists of a noun stem 'vayathu' (age) and the suffix
 'aana'. 
 
Derived adjectives are usually obtained by adding suffixes like 'aana' etc to nouns.
\end{enumerate}
 
\section{Initial Design}

To perform the annotation, I have tried to follow the rules that were discussed in the annotation manual by Ramaswamy et al (\cite{ramReport}).
The manual consists of several example sentences that cover a wide range of phenomena which was useful for me in making
some decisions. In most of the examples that I annotated, the head word mostly occured at the end of the sentence. The annotated corpora A anb B have been uploaded in the repository. Each of them contain around 1000 tokens. Currently, I have just added the head information for each term
in CONLL format. I have also included the text of the string in addition to the annotations. But as of now, I have not extracted any features such as 
POS tags and morphological information to the output. I could find only one open source POS tagger for Tamil (\cite{dhanalakshmi2010natural}) and I was unable to get the software
to work. But by the next deadline, I think that I would be able to use the POS tag features as input for my model. Also, I am aware that features based
on morphological analysis would be immensely useful for highly inflected languages such as Tamil, but again I was unable to get the morphological 
analyzer that was built by the same group (\cite{dhanalakshmi2010natural}) to work. I hope to resolve some dependency issues in the software to extract
POS tags and morphological features before building my model. My idea is to build a supervised dependency parser by annotating some additional training data. 
I feel that supervised approaches would work well because I think more training data would help capture the wider range of morphological phenomena that occurs in Tamil words. 
I am currently annotating more data for the supervised model and I plan to use MaltParser for training the model. My eventual goal would be
to build a stable data driven supervised dependency parser guided by a rich feature set such as POS tags and morphological annotations. 

\section{System Analysis on Corpus A}

\section{Lessons Learned and Revised Design}

\section{System Analysis on Corpus B}

\section{Final Revisions}

\section{Future Work}

\bibliographystyle{plainnat}
\bibliography{pradeep-report}

\label{lastpage}
\end{document}
